\documentclass[a4paper,11pt]{texMemo}
\usepackage[english]{babel}
\usepackage{graphicx}
\usepackage{url}
\usepackage[dvipsnames,svgnames,x11names]{xcolor}

%%%%%%%%%%%%%%%%%%% tracking changes %%%%%%%%%%%%%%%%%%
\usepackage[markup=underlined]{changes}
% remarks in the margins instead of footnotes
\usepackage{todonotes}
\setlength{\marginparwidth}{2cm}
\makeatletter
\setremarkmarkup{\todo[color=Changes@Color#1!20,size=\scriptsize]{#1: #2}}
\makeatother
\newcommand{\note}[2][]{\added[#1,remark={#2}]{}}
% define reviewer
\definechangesauthor[color=blue]{RP}

% TO REMOVE ALL TRACKING MARKUPS:
% 1. remove comment on lines 22 and 23
% 2. add comment on lines 8,10,13,17
%\usepackage[final]{changes}
%\usepackage[disable]{todonotes}
%%%%%%%%%%%%%%%%%%%%%%%%%%%%%%%%%%%%%%%%%%%%%%%%%%%%%%%

\memoto{Organizing Committee, Atlantic Causal Inference Conference 2019}
\memofrom{Prof.~Mark J.~van der Laan}
\memosubject{Workshop proposal: The \texttt{tlverse} software ecosystem for
  causal inference.}
\memodate{\today}
\logo{\includegraphics[scale=0.15]{figs/ucberkeleyseal_874_540.eps}}

\begin{document}
\maketitle
\vspace{-0.25in}
\section{Workshop information}

\subsection{\textbf{Title:} \small The \texttt{tlverse}: A Software Ecosystem
  for Causal Inference and Targeted Learning}

\subsection{Abstract}

This full-day workshop will provide a comprehensive introduction to both the
\texttt{tlverse} software ecosystem and the field of targeted learning for
causal inference. While this will primarily be a software workshop centered
around the new \texttt{tlverse} ecosystem (\url{https://github.com/tlverse}) of
\texttt{R} packages, there will be rigorous examination of both causal inference
methodology --- focusing on the field of targeted learning --- and applications
in both large-scale observational studies and randomized experiments. The focus
will be on introducing modern methodological developments in statistical causal
inference and their corresponding software implementations in the
\texttt{tlverse}. Through vignette-guided live coding exercises, participants
will perform hands-on implementation of novel estimators for assessing causal
claims with complex, observational data. Topics to be addressed include ensemble
machine learning; efficient substitution estimators in nonparametric and
semiparametric models and targeted minimum loss-based estimation; inference
based on influence functions; static, dynamic, optimal dynamic, and stochastic
interventions. Strategies for constructing (double) robust efficient plug-in
estimators with normal limiting distributions, allowing for valid inference even
when the functional nuisance parameters are estimated via machine learning.
Causal parameters and and corresponding estimators will be examined both
mathematically and through their corresponding \texttt{R} package
implementations from the \texttt{tlverse} ecosystem via hands-on data analysis,
providing participants opportunities to develop skills that will translate to
real-world causal inference analyses. Some background in mathematical statistics
will be useful; familiarity with the \texttt{R} programming language will be
essential.


\subsection{Motivation}

Randomized clinical trials (RCTs) have long served as the gold standard of
evidence for comparing potential interventions in clinical medicine, public
health, and marketing, political science, and many other fields. Unfortunately,
such trials are often not feasible due to ethical, logistic or economical
constraints. Observational studies constitute a potentially rich alternative to
RCTs, providing an opportunity to learn about the causal effects of
interventions for which little or no trial data can be produced; however, in
such studies intervention allocation may be strongly confounded by other
important characteristics. Thus, great care is needed in attempts to disentangle
observed relationships and, ultimately, infer causal effects. This workshop will
provide a comprehensive introduction to the field of targeted learning, a modern
statistical framework that utilizes state-of-the-art machine learning to
flexibly adjust for confounding while yielding efficient, unbiased estimators
and valid statistical inference, thus unlocking observational studies for causal
inference.

Targeted learning is a complex statistical approach and, in order for this
method to be accessible in practice, it is crucial that it is accompanied by
robust software. The \texttt{tlverse} software ecosystem was developed to
fulfill this need. Not only does this software facilitate computationally
reproducible and efficient analyses, but it is also a tool for targeted learning
education since its workflow mirrors that of the methodology. That is, the
\texttt{tlverse} paradigm does not focus on implementing a specific estimator or
a small set of related estimators --- instead, the focus is on exposing the
statistical framework of targeted learning itself! Thus, users are required to
explicitly define objects to model key statistical objects: the nonparametric
structural equation model, the factorized likelihood, counterfactual
interventions, causal parameters, and algorithmic step for computing estimators.
All \texttt{R} packages in the \texttt{tlverse} ecosystem directly model the key
objects defined in the mathematical and theoretical framework of targeted
learning. What's more, the \texttt{tlverse} \texttt{R} packages share a core set
of design principles centered on extensibility, allowing for them to be used in
conjuction with each other and built upon one other in a cohesive fashion.

\subsection{Duration}

This will be a full-day workshop, featuring modules that introduce a distinct
causal question motivated by a case study, alongside statistical methodology and
software for assessing the causal claim of interest. A sample schedule would
take the form:
\begin{itemize}
  \itemsep0pt
  \item 09:30AM--10:15AM: Introduction to targeted learning for causal inference
  \item 10:15AM--10:45AM: Introduction to the \texttt{tlverse} software
    ecosystem
  \item 10:45AM--11:00AM: Coffee Break
  \item 11:00AM--11:45AM: Ensemble machine learning with the \texttt{sl3}
    package
  \item 11:45AM--12:30PM: Targeted learning for causal inference with the
    \texttt{tmle3} package
  \item 12:30PM--1:30PM: Lunch
  \item 01:30PM--02:45PM: Optimal treatments regimes and the
    \texttt{tmle3mopttx} package
  \item 02:45PM--03:00PM: Coffee Break
  \item 3:00PM--4:00PM: Stochastic interventions and the \texttt{tmle3shift}
    package
  \item 04:00PM--4:30PM: Course summary and concluding remarks
\end{itemize}
\textit{Please note that the workshop will be 6 hours, including coffee breaks
but not lunch.}

\subsection{Prior History}

The \texttt{tlverse} ecosystem is a relatively recent effort, about 2 years in
the making. Although some material has been introduced across several
graduate-level courses taught at UC Berkeley, this workshop would be the first
offering in the 6-hour format.

\section{Organizers}

\subsection*{Mark van der Laan, Ph.D.}

\vspace{-.5em}

Mark van der Laan, PhD, is Professor of Biostatistics and Statistics at UC
Berkeley. His research interests include statistical methods in computational
biology, survival analysis, censored data, adaptive designs, targeted maximum
likelihood estimation, causal inference, data-adaptive loss-based learning, and
multiple testing. His research group developed loss-based super learning in
semiparametric models, based on cross-validation, as a generic optimal tool for
the estimation of infinite-dimensional parameters, such as nonparametric density
estimation and prediction with both censored and uncensored data. Building on
this work, his research group developed targeted maximum likelihood estimation
for a target parameter of the data-generating distribution in arbitrary
semiparametric and nonparametric models, as a generic optimal methodology for
statistical and causal inference. Most recently, Mark's group has focused in
part on the development of a centralized, principled set of software tools for
targeted learning, the \texttt{tlverse}. Contact: \texttt{laan@berkeley.edu}.

\vspace{-.5em}

\subsection*{Alan Hubbard, Ph.D.}

\vspace{-.5em}

Alan Hubbard is Professor of Biostatistics, former head of the Division of
Biostatistics at UC Berkeley, and head of data analytics core at UC Berkeley's
SuperFund research program. His current research interests include causal
inference, variable importance analysis, statistical machine learning,
estimation of and inference for data-adaptive statistical target parameters, and
targeted minimum loss-based estimation. Research in his group is generally
motivated by applications to problems in computational biology, epidemiology,
and precision medicine. Contact: \texttt{hubbard@berkeley.edu}.

\vspace{-.5em}

\subsection*{Jeremy Coyle, Ph.D.}

\vspace{-.5em}

Jeremy Coyle is a consulting data scientist and statistical programmer,
currently leading the software development effort that has produced the
\texttt{tlverse} ecosystem of \texttt{R} packages and related software tools.
Jeremy earned his PhD in Biostatistics from UC Berkeley in 2016, primarily under
the supervision of Alan Hubbard. Contact: \texttt{jeremyrcoyle@gmail.com}.

\vspace{-.5em}

\subsection*{Nima Hejazi, M.A.}

\vspace{-.5em}

Nima is a PhD candidate in biostatistics with a designated emphasis in
computational and genomic biology, working jointly with Mark van der Laan and
Alan Hubbard. Nima is affiliated with UC Berkeley's Center for Computational
Biology and NIH Biomedical Big Data training program. His research interests
span causal inference, nonparametric inference and machine learning, targeted
loss-based estimation, survival analysis, statistical computing, reproducible
research, and high-dimensional biology. He is also passionate about software
development for applied statistics, including software design, automated
testing, and reproducible coding practices. Contact:
\texttt{nhejazi@berkeley.edu}.

\vspace{-.5em}

\subsection*{Ivana Malenica, M.A.}

\vspace{-.5em}

Ivana is a Ph.D.~student in biostatistics advised by Mark van der Laan. Ivana is
currently a fellow at the Berkeley Institute for Data Science, after serving as
a NIH Biomedical Big Data and Freeport-McMoRan Genomic Engine fellow. She earned
her Master's in Biostatistics and Bachelor's in Mathematics, and spent some time
at the Translational Genomics Research Institute. Very broadly, her research
interests span non/semi-parametric theory, probability theory, machine learning,
causal inference and high-dimensional statistics. Most of her current work
involves complex dependent settings (dependence through time and network) and
adaptive sequential designs. Contact: \texttt{imalenica@berkeley.edu}.

\vspace{-.5em}

\subsection*{Rachael Phillips, M.A.}

\vspace{-.5em}

Rachael is a Ph.D.~student in biostatistics, advised by Alan Hubbard and Mark
van der Laan. She has an M.A.~in Biostatistics, B.S.~in Biology with a
Chemistry minor and a B.A.~in Mathematics with a Spanish minor. Rachael is
motivated to solve real-world, high-dimensional problems in human health. Her
research interests span causal inference, machine learning, nonparametric
statistical estimation, and finite sample inference. She is also passionate
about online mediated education. Rachael is affiliated with UC Berkeley's Center
for Computational Biology, NIH Biomedical Big Data Training Program, and
Superfund Research Program. Contact: \texttt{rachaelvphillips@berkeley.edu}.

\end{document}
